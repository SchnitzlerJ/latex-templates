\documentclass{scrartcl}

\usepackage{xcolor}

\usepackage{float}

\usepackage{tikz}
\usepackage{pgfplots}
\usetikzlibrary{iswplots}
\usepackage{tikzscale}

\usepackage{relsize}

\usepackage{filecontents}

\usepackage{minted}
\setminted{
  numberblanklines=true,%
  breaklines=true,%
  fontsize=\smaller,%
}
\newmintinline{latex}{}
\newmintinline{text}{}

\usepackage{cleveref}


\begin{filecontents}{iswplots_sample1.tikz}
\begin{tikzpicture}
  \begin{axis}[
    % iswlineplot,% basic style that needs to be applied to the axis environment
    domain=-5:5,% domain of x reaching from -5 to 5
    samples=100,% set number of samples to use for plotting data
  ]

    \addplot {sin(deg(x))};
    \addlegendentry{sin(deg(x))};

    \addplot {sin(deg(2*x))};
    \addlegendentry{sin(deg(2*x))};

  \end{axis}
\end{tikzpicture}
\end{filecontents}


\begin{filecontents}{iswplots_sample2.tikz}
\begin{tikzpicture}
  \begin{axis}[
    iswlineplot,% basic style that needs to be applied to the axis environment
    domain=-5:5,% domain of x reaching from -5 to 5
    samples=100,% set number of samples to use for plotting data
  ]

    \addplot {sin(deg(x - 0/7*pi))};
    \addlegendentry{sin(deg(x - 0/7*pi))};

    \addplot {sin(deg(x - 1/7*pi))};
    \addlegendentry{sin(deg(x - 1/7*pi))};

    \addplot {sin(deg(x - 2/7*pi))};
    \addlegendentry{sin(deg(x - 2/7*pi))};

    \addplot {sin(deg(x - 3/7*pi))};
    \addlegendentry{sin(deg(x - 3/7*pi))};

    \addplot {sin(deg(x - 4/7*pi))};
    \addlegendentry{sin(deg(x - 4/7*pi))};

    \addplot {sin(deg(x - 5/7*pi))};
    \addlegendentry{sin(deg(x - 5/7*pi))};

    \addplot {sin(deg(x - 6/7*pi))};
    \addlegendentry{sin(deg(x - 6/7*pi))};

    \addplot {sin(deg(x - 7/7*pi))};
    \addlegendentry{sin(deg(x - 7/7*pi))};

    \addplot {sin(deg(x - 8/7*pi))};
    \addlegendentry{sin(deg(x - 8/7*pi))};

    \addplot {sin(deg(x - 9/7*pi))};
    \addlegendentry{sin(deg(x - 9/7*pi))};

    \addplot {sin(deg(x - 10/7*pi))};
    \addlegendentry{sin(deg(x - 10/7*pi))};

  \end{axis}
\end{tikzpicture}
\end{filecontents}


\begin{filecontents}{iswplots_sample3.tikz}
\begin{tikzpicture}
  \begin{axis}[
    centered iswlineplot,%
    domain=-5:5,% domain of x reaching from -5 to 5
    samples=100,% set number of samples to use for plotting data
  ]

    \addplot {sin(deg(x))};
    \addlegendentry{sin(deg(x))};

    % Change the next plot to only have 50 samples
    \addplot+[samples=50] {sin(deg(2*x))};
    \addlegendentry{sin(deg(2*x))};

  \end{axis}
\end{tikzpicture}
\end{filecontents}


\begin{filecontents}{iswplots_sample4.tikz}
\begin{tikzpicture}
  \begin{axis}[
    iswbarplot,%
    xlabel={Foo},
    ylabel={Bar},
  ]

    \addplot plot coordinates {(1, 20) (2, 25) (3, 22.4) (4, 12.4)};
    \addlegendentry{lorem};
    \addplot plot coordinates {(1, 18) (2, 24) (3, 23.5) (4, 13.2)};
    \addlegendentry{ipsum};
    \addplot plot coordinates {(1, 10) (2, 19) (3, 25) (4, 15.2)};
    \addlegendentry{dolor};

    % \legend{lorem, ipsum, color}

  \end{axis}
\end{tikzpicture}
\end{filecontents}





\begin{document}

% \begin{figure}

%   % \caption{%
%   %   Axis with style \latexinline{iswlineplot}.%
%   % }
% \end{figure}


\section{Line Plots}


\subsection{Standard Line Plots}

In order to make a tikzpicture with axes confine with the style guide, the encapsulated \latexinline{axis}-environment must be explicitely given the \latexinline{iswlineplot} option.

\begin{listing}[H]
  \inputminted{latex}{iswplots_sample1.tikz}
  \caption{Code sample for the figure shown in \cref{fig:iswplots_sample1}}
  \label{lst:iswplots_sample1}
\end{listing}

\begin{figure}[H]
  \centering
  \smaller
  \includegraphics[height=6cm, width=0.9\linewidth]{iswplots_sample1}
  \caption{%
    Figure generated by code sample from \cref{lst:iswplots_sample1}
  }
  \label{fig:iswplots_sample1}
\end{figure}


\subsection{Different Line Styles}



\begin{listing}[H]
  \inputminted{latex}{iswplots_sample2.tikz}
  \caption{Code sample for the figure shown in \cref{fig:iswplots_sample2}}
  \label{lst:iswplots_sample2}
\end{listing}

\begin{figure}[H]
  \centering
  \smaller
  \includegraphics[height=6cm, width=0.9\linewidth]{iswplots_sample2}
  \caption{%
    Figure generated by code sample from \cref{lst:iswplots_sample2}
  }
  \label{fig:iswplots_sample2}
\end{figure}


\subsection{Centered Line Plots}

Line plots can be also have centered coordinate axes.
This might be helpful in case one wants to mimic an oscilloscope or the like.

\begin{listing}[H]
  \inputminted{latex}{iswplots_sample3.tikz}
  \caption{Code sample for the figure shown in \cref{fig:iswplots_sample3}}
  \label{lst:iswplots_sample3}
\end{listing}

\begin{figure}[H]
  \centering
  \smaller
  \includegraphics[height=6cm, width=0.9\linewidth]{iswplots_sample3}
  \caption{%
    Figure generated by code sample from \cref{lst:iswplots_sample3}
  }
  \label{fig:iswplots_sample3}
\end{figure}




\paragraph{Change Marker Size}
The default marker size of \latexinline{0.45ex} can be changed by setting it in an appropiate \latexinline{tikzset} command like \latexinline{\tikzset{isw marker size=0.45ex}}.


\paragraph{Increase Number of Evenly Spaced Markers}
By default, 10 segments are put onto a line plot causing 11 markers to be drawn with the first and last marker on the first and last piece of the line.
The markers are evenly spaced along the path and \emph{not along the abscissa}.
To change the number of markers, set the appropiate TikZ config using \latexinline{\tikzset{isw marker count=50}}.


\section{Bar Plots}

\begin{listing}[H]
  \inputminted{latex}{iswplots_sample4.tikz}
  \caption{Code sample for the figure shown in \cref{fig:iswplots_sample4}}
  \label{lst:iswplots_sample4}
\end{listing}

\begin{figure}[H]
  \centering
  \smaller
  \includegraphics[height=6cm, width=0.9\linewidth]{iswplots_sample4}
  \caption{%
    Figure generated by code sample from \cref{lst:iswplots_sample4}
  }
  \label{fig:iswplots_sample4}
\end{figure}


\end{document}
